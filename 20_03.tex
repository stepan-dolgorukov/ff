\documentclass[14pt,a4paper,oneside,russian]{article}

\usepackage[T2A]{fontenc}
\usepackage[russian]{babel}
\usepackage{amsmath}
\usepackage{amsthm}

\newtheorem*{axiom1}{Ассоциативность сложения (сочетательный закон)}
\newtheorem*{axiom2}{Существование нейтрального элемента по сложению}
\newtheorem*{axiom3}{Существование обратного элемента по сложению}
\newtheorem*{axiom4}{Коммутативность сложения (перестановочный закон)}
\newtheorem*{axiom5}{Дистрибутивность умножения относительно сложения (распределительный закон)}

\begin{document}

\begin{axiom1}
$$ ((a_1, b_1) + (a_2, b_2)) + (a_3, b_3) = $$
$$ = (\frac{a_1}{b_1} + \frac{a_2}{b_2}) + \frac{a_3}{b_3} = $$
$$ = \frac{a_1 * b_2 + a_2 * b_1}{b_1 * b_2} + \frac{a_3}{b_3} = $$
$$ = \frac{(a_1 * b_2 + a_2 * b_1) * b_3 + a_3 * (b_1 * b_2)}{b_1 * b_2 * b_3} = $$
$$ = \frac{a_1 * b_2 * b_3 + a_2 * b_1 * b_3 + a_3 * b_1 * b_2}{b_1 * b_2 * b_3} $$

Сравним с:
$$ ( a_1, b_1 ) + (( a_2, b_2 ) + ( a_3, b_3 )) = $$
$$ = \frac{a_1}{b_1} + (\frac{a_2}{b_2} + \frac{a_3}{b_3}) = $$
$$ = \frac{a_1}{b_1} + (\frac{a_2 * b_3 + a_3 * b_2}{b_2 * b_3}) = $$
$$ = \frac{(b_2 * b_3) * a_1 + (a_2 * b_3 + a_3 * b_2) * b_1}{b_1 * b_2 * b_3} = $$
$$ = \frac{b_2 * b_3 * a_1 + a_2 * b_3 * b_1 + a_3 * b_2 * b_1}{b_1 * b_2 * b_3} =  $$

Пользуемся коммутативностью (перестановочным законом) вещественных чисел:

$$ = \frac{a_1 * b_2 * b_3 + a_2 * b_1 * b_3 + a_3 * b_1 * b_2}{b_1 * b_2 * b_3} $$

Получились одинаковые формулы. $ \implies $ Они задают равные элементы. $ \implies $
$ ((a_1, b_1) + (a_2, b_2)) + (a_3, b_3) = (a_1, b_1) + ((a_2, b_2) + (a_3, b_3)) $

\end{axiom1}
\newpage

\begin{axiom2}
$$ (a, b) + (0, 1) = (a * 1 + b * 0, b * 1) = (a, b) $$
\begin{center}
$ \implies (0, 1) $ - нейтральный по сложению справа (1)
\end{center}
$$ (0, 1) + (a, b) = (0 * b + 1 * a, 1 * b) = (a, b) $$
\begin{center}
$ \implies (0, 1) $ - нейтральный по сложению справа (1)
\end{center}
\end{axiom2}

$ (1), (2) \implies (0, 1) $ - нейтральный элемент по сложению
\newpage

\begin{axiom3}
$$ (a, b) + (a, -b) = (a * (-b) + b * a, b * (-b)) = $$
$$ = ( -(a * b) + (a * b), b * (-b) ) = (0, b * (-b)) = (0, 1) $$
$ \implies (a, -b) = -(a, b) $

$$ (a, b) + (-a, b) = (a * b + b * (-a), b * b) = $$
$$ = (a * b + ((-a) * b), b * b) = (0, b * b) = (0, 1) $$
$ \implies -(a, b) = (-a, b) $
\end{axiom3}
\newpage

\begin{axiom4}
$$ (a, b) + (c , d) = \frac{a}{b} + \frac{c}{d} = \frac{a * d + c * b}{b * d} $$
$$ (c, d) + (a, b) = \frac{c}{d} + \frac{a}{b} = \frac{c * b + d * a}{d * b} = $$

Пользуемся коммутативностью сложения вещественных чисел:
$$ = \frac{c * b + d * a}{d * b} = \frac{a * d + b * c}{b * d} $$
Получили такую же формулу, что и в сумме $ (a, b) + (c,d) \implies (a, b) + (c,d) = (c, d) + (a, b). $
\end{axiom4}
\newpage

\begin{axiom5}

$$ \newline $$
Дистрибутивность слева:
$$ ((a_1, b_1) + (a_2, b_2)) * (a_3, b_3) = (\frac{a_1}{b_1} + \frac{a_2}{b_2}) * \frac{a_3}{b_3} = $$
$$ = (\frac{a_1 * b_2 + b_1 * a_2}{b_1 * b_3}) * \frac{a_3}{b_3} =  $$
$$ = \frac{a_1 * b_2 * a_3 + b_1 * a_2 * a_3}{b_1 * b_2 * b_3} =  $$
$$ = \frac{a_1 * b_2 * a_3}{b_1 * b_2 * b_3} + \frac{b_1 * a_2 * a_3}{b_1 * b_2 * b_3} =  $$
$$ = \frac{a_1 * a_3}{b_1 * b_3} + \frac{a_2 * a_3}{b_2 * b_3} =  $$
$$ = \frac{a_1}{b_1} * \frac{a_3}{b_3} + \frac{a_2}{b_2} * \frac{a_3}{a_3} =  $$
$$ \newline $$
Дистрибутивность справа:
$$
(a_3, b_3) * ((a_1, b_1) + (a_2, b_2)) =
\frac{a_3}{b_3} * (\frac{a_1 * b_2 + b_1 * a_2}{b_1 * b_2}) =
$$
$$
= \frac{a_3 * a_1 * b_2}{b_3 * b_1 * b_2} + \frac{a_3 * b_1 * a_2}{b_3 * b_1 * b_2}
= \frac{a_3 * a_1}{b_3 * b_1} + \frac{a_3 * a_2}{b_3 * b_2} =
$$
$$
= \frac{a_3}{b_3} * \frac{a_1}{b_1} + \frac{a_3}{b_3} * \frac{a_2}{b_2}
$$
\end{axiom5}

\end{document}
