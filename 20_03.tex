\documentclass[14pt,a4paper,oneside,russian]{article}

\usepackage[T2A]{fontenc}
\usepackage[russian]{babel}
\usepackage{amsmath}
\usepackage{amsthm}

\newtheorem*{axiom1}{Ассоциативность сложения (сочетательный закон)}
\newtheorem*{axiom2}{Существование нейтрального элемента по сложению}
\newtheorem*{axiom3}{Существование обратного элемента по сложению}
\newtheorem*{axiom4}{Коммутативность сложения (перестановочный закон)}
\newtheorem*{axiom5}{Дистрибутивность умножения относительно сложения (распределительный закон)}
\newtheorem*{axiom6}{Ассоциативность умножения (сочетательный закон)}
\newtheorem*{axiom7}{Коммутативность умножения (перестановочный закон)}
\newtheorem*{axiom8}{Существование нейтрального по умножению}
\newtheorem*{axiom9}{Существование обратного по умножению у ненулевого}

\begin{document}

\begin{axiom1}
$$
((a_1, b_1) + (a_2, b_2)) + (a_3, b_3) =
(\frac{a_1}{b_1} + \frac{a_2}{b_2}) + \frac{a_3}{b_3} =
\frac{a_1 * b_2 + a_2 * b_1}{b_1 * b_2} + \frac{a_3}{b_3} =
$$
$$
\frac{(a_1 * b_2 + a_2 * b_1) * b_3 + a_3 * (b_1 * b_2)}{b_1 * b_2 * b_3} =
\frac{a_1 * b_2 * b_3 + a_2 * b_1 * b_3 + a_3 * b_1 * b_2}{b_1 * b_2 * b_3}
$$

Сравним с:
$$
( a_1, b_1 ) + (( a_2, b_2 ) + ( a_3, b_3 )) =
\frac{a_1}{b_1} + (\frac{a_2}{b_2} + \frac{a_3}{b_3}) =
\frac{a_1}{b_1} + (\frac{a_2 * b_3 + a_3 * b_2}{b_2 * b_3}) =
$$
$$
= \frac{(b_2 * b_3) * a_1 + (a_2 * b_3 + a_3 * b_2) * b_1}{b_1 * b_2 * b_3} =
\frac{b_2 * b_3 * a_1 + a_2 * b_3 * b_1 + a_3 * b_2 * b_1}{b_1 * b_2 * b_3} =
$$

Пользуемся коммутативностью (перестановочным законом) вещественных чисел:

$$
= \frac{a_1 * b_2 * b_3 + a_2 * b_1 * b_3 + a_3 * b_1 * b_2}{b_1 * b_2 * b_3}
$$

Получились одинаковые формулы.
$ \implies $
Они задают равные элементы.
$ \implies $
$
((a_1, b_1) + (a_2, b_2)) + (a_3, b_3) =
(a_1, b_1) + ((a_2, b_2) + (a_3, b_3))
\newline
$

Теперь с классами.
$$
(\overline{(a_1, b_1)} + \overline{(a_2, b_2)}) + \overline{(a_3, b_3)} =
(\overline{(a_1 * b_2 + b_1 * a_2, b_1 * b_2)} + \overline{(a_3, b_3)} =
$$
$$
=\overline{(a_1 * b_2 * b_3 + b_1 * a_2 * b_3 + a_3 * b_1 * b_2,
b_1 * b_2 * b_3)} =
$$
$$
= \overline{(a_1 * b_2 * b_3 + a_2 * b_1 * b_3 + a_3 * b_1 * b_2,
b_1 * b_2 * b_3)}
$$

$$
\overline{(a_1, b_1)} + (\overline{(a_2, b_2)} + \overline{(a_3, b_3)}) =
\overline{(a_1, b_1)} + (\overline{(a_2 * b_3 + b_2 * a_3, b_2 * b_3)} =
$$
$$
= \overline{(a_1 * b_2 * b_3 + b_1 * a_2 * b_3 + b_1 * b_2 * a_3,
b_1 * b_2 * b_3)} =
$$
$$
\overline{(a_1 * b_2 * b_3 + a_2 * b_1 * b_3 + a_3 * b_1 * b_2,
b_1 * b_2 * b_3)}
$$
\end{axiom1}
\newpage

\begin{axiom2}
$$ (a, b) + (0, 1) = (a * 1 + b * 0, b * 1) = (a, b) $$
$$
(a, b) + (0, 1) =
\frac{a}{b} + \frac{0}{1} =
\frac{a * 1 + b * 0}{b * 1} =
\frac{a + 0}{b} =
\frac{a}{b}
$$

$ \implies (0, 1) $ - нейтральный по сложению справа (1)

$$ (0, 1) + (a, b) = (0 * b + 1 * a, 1 * b) = (a, b) $$
$$
(0, 1) + (a, b) =
\frac{0}{1} + \frac{a}{b} =
\frac{0 * b + 1 * a}{1 * b} =
\frac{0 + a}{b} =
\frac{a}{b}
$$

$ \implies (0, 1) $ - нейтральный по сложению справа (2)

$ \newline (1), (2) \implies (0, 1) $ - нейтральный элемент по сложению

$$
\overline{(0, 1)} + \overline{(a, b)} =
\overline{(0 * b + 1 * a, 1 * b)} =
\overline{(0 + a, b)} =
\overline{(a, b)}
$$
$$
\overline{(a, b)} + \overline{(0, 1)} =
\overline{(a * 1 + b * 0, b * 1)} =
\overline{(a + 0, b)} =
\overline{(a, b)}
$$

\end{axiom2}
\newpage

\begin{axiom3}
$$
(a, b) + (a, -b) =
(a * (-b) + b * a, b * (-b)) =
$$
$$
= (-(a * b) + (a * b), b * (-b)) =
(0, b * (-b)) \sim (0, 1)
$$
$ \implies (a, -b) = -(a, b) $

$$
(a, b) + (-a, b) =
(a * b + b * (-a), b * b) =
$$
$$
= (a * b + ((-a) * b), b * b)
= (0, b * b) \sim (0, 1)
$$
$ \implies -(a, b) = (-a, b) $

$$
\overline{(a, b)} + \overline{(-a, b)} =
\overline{(a * b + b * (-a), b * b)} =
$$
$$
\overline{((a * b) + (-(a * b)), b * b)} =
\overline{(0, b * b)} =
\overline{(0, 1})
$$

$$
\overline{(a, b)} + \overline{(a, -b)} =
\overline{(a * (-b) + b * a, b * (-b)} =
$$
$$
\overline{(-(a * b) + (a * b), b * (-b)} =
\overline{(0, b * (-b))} =
\overline{(0, 1})
$$
\end{axiom3}
\newpage

\begin{axiom4}
$$ (a, b) + (c , d) = \frac{a}{b} + \frac{c}{d} = \frac{a * d + c * b}{b * d} $$
$$ (c, d) + (a, b) = \frac{c}{d} + \frac{a}{b} = \frac{c * b + d * a}{d * b} = $$

Пользуемся коммутативностью сложения вещественных чисел:
$$ = \frac{c * b + d * a}{d * b} = \frac{a * d + b * c}{b * d} $$
Получили такую же формулу, что и в сумме $ (a, b) + (c,d) \implies (a, b) + (c,d) = (c, d) + (a, b). $

\[
\overline{(c,d)} + \overline{(a,b)} \overset{def}{=}
\overline{(c*b+d*a,d*b)} \overset{\text{Комм. * и + вещ.чисел}}{=}
\overline{(a*d+b*c,b*d)} \overset{def}{=}
\]
\[
\overset{def}{=} \overline{(a,b)} + \overline{(c,d)}
\]
\end{axiom4}
\newpage

\begin{axiom5}

$$ \newline $$
Дистрибутивность слева:
$$ ((a_1, b_1) + (a_2, b_2)) * (a_3, b_3) = (\frac{a_1}{b_1} + \frac{a_2}{b_2}) * \frac{a_3}{b_3} = $$
$$ = (\frac{a_1 * b_2 + b_1 * a_2}{b_1 * b_3}) * \frac{a_3}{b_3} =  $$
$$ = \frac{a_1 * b_2 * a_3 + b_1 * a_2 * a_3}{b_1 * b_2 * b_3} =  $$
$$ = \frac{a_1 * b_2 * a_3}{b_1 * b_2 * b_3} + \frac{b_1 * a_2 * a_3}{b_1 * b_2 * b_3} =  $$
$$ = \frac{a_1 * a_3}{b_1 * b_3} + \frac{a_2 * a_3}{b_2 * b_3} =  $$
$$ = \frac{a_1}{b_1} * \frac{a_3}{b_3} + \frac{a_2}{b_2} * \frac{a_3}{a_3} =  $$
$$ \newline $$
Дистрибутивность справа:
$$
(a_3, b_3) * ((a_1, b_1) + (a_2, b_2)) =
\frac{a_3}{b_3} * (\frac{a_1 * b_2 + b_1 * a_2}{b_1 * b_2}) =
$$
$$
= \frac{a_3 * a_1 * b_2}{b_3 * b_1 * b_2} + \frac{a_3 * b_1 * a_2}{b_3 * b_1 * b_2}
= \frac{a_3 * a_1}{b_3 * b_1} + \frac{a_3 * a_2}{b_3 * b_2} =
$$
$$
= \frac{a_3}{b_3} * \frac{a_1}{b_1} + \frac{a_3}{b_3} * \frac{a_2}{b_2}
$$
\end{axiom5}
\newpage

\begin{axiom6}
$$ ((a_1, b_1) * (a_2, b_2)) * (a_3, b_3) = (\frac{a_1}{b_1} * \frac{a_2}{b_2}) * \frac{a_3}{b_3} = $$
$$ = \frac{a_1 * a_2}{b_1 * b_2} * \frac{a_3}{b_3} = \frac{a_1 * a_2 * a_3}{b_1 * b_2 * b_3} $$

$$ (a_1, b_1 * ((a_2, b_2)) * (a_3, b_3)) = \frac{a_1}{b_1} * (\frac{a_2}{b_2} * \frac{a_3}{b_3}) = $$
$$ = \frac{a_1}{b_1} * \frac{a_2 * a_3}{b_2 * b_3} = \frac{a_1 * a_2 * a_3}{b_1 * b_2 * b_3} $$
\end{axiom6}
\newpage

\begin{axiom7}
$$ (a_1, b_1) * (a_2, b_2) = \frac{a_1}{b_1} * \frac{a_2}{b_2} = \frac{a_1 * a_2}{b_1 * b_2} $$
$$ (a_2, b_2) * (a_1, b_1) = \frac{a_2}{b_2} * \frac{a_1}{b_1} = \frac{a_2 * a_1}{b_2 * b_1} $$

$$ a_1 * a_2 = a_2 * a_1, b_1 * b_2 = b_1 * b_2 \implies \frac{a_1 * a_2}{b_1 * b_2} = \frac{a_2 * a_1}{b_2 * b_1} $$
\end{axiom7}
\newpage

\begin{axiom8}
$$ (a, b) * (1, 1) = (a * 1, b * 1) = (a, b) $$
$$
(a, b) * (1, 1) = \frac{a}{b} * \frac{1}{1} = \frac{a * 1}{b * 1} = \frac{a}{b}
\newline
$$
$ \implies (1, 1) = \frac{1}{1} $ - нейтральный элемент по умножению слева (1)

$$ (1, 1) * (a, b) = (1 * a, 1 * b) = (a, b) $$
$$
(1, 1) * (a, b) = \frac{1}{1} * \frac{a}{b} = \frac{1 * a}{1 * b} = \frac{a}{b}
\newline
$$
$ \implies (1, 1) = \frac{1}{1} $ - нейтральный элемент по умножению справа (2)

$ \newline (1), (2) \implies (1, 1) $ - нейтральный элемент по умножению.
\end{axiom8}
\newpage

\begin{axiom9}
$$ (a, b) \neq (0, x) $$
$$ (a, b)^{-1} = (b, a) $$
$$ (a, b) * (b, a) = \frac{a}{b} * \frac{b}{a} = \frac{a * b}{b * a} = \frac{1}{1} $$
\end{axiom9}

\end{document}
